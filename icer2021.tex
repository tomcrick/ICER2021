%%
%% This is file `sample-manuscript.tex',
%% generated with the docstrip utility.
%%
%% The original source files were:
%%
%% samples.dtx  (with options: `manuscript')
%% 
%% IMPORTANT NOTICE:
%% 
%% For the copyright see the source file.
%% 
%% Any modified versions of this file must be renamed
%% with new filenames distinct from sample-manuscript.tex.
%% 
%% For distribution of the original source see the terms
%% for copying and modification in the file samples.dtx.
%% 
%% This generated file may be distributed as long as the
%% original source files, as listed above, are part of the
%% same distribution. (The sources need not necessarily be
%% in the same archive or directory.)
%%
%%
%% Commands for TeXCount
%TC:macro \cite [option:text,text]
%TC:macro \citep [option:text,text]
%TC:macro \citet [option:text,text]
%TC:envir table 0 1
%TC:envir table* 0 1
%TC:envir tabular [ignore] word
%TC:envir displaymath 0 word
%TC:envir math 0 word
%TC:envir comment 0 0
%%
%%
%% The first command in your LaTeX source must be the \documentclass command.
\documentclass[manuscript,review]{acmart}

%%
%% \BibTeX command to typeset BibTeX logo in the docs
\AtBeginDocument{%
  \providecommand\BibTeX{{%
    \normalfont B\kern-0.5em{\scshape i\kern-0.25em b}\kern-0.8em\TeX}}}

%% Rights management information.  This information is sent to you
%% when you complete the rights form.  These commands have SAMPLE
%% values in them; it is your responsibility as an author to replace
%% the commands and values with those provided to you when you
%% complete the rights form.
% \setcopyright{acmcopyright}
% \copyrightyear{2021}
% \acmYear{2021}
% \acmDOI{}

%% These commands are for a PROCEEDINGS abstract or paper.
% \acmConference[Woodstock '18]{Woodstock '18: ACM Symposium on Neural
%   Gaze Detection}{June 03--05, 2018}{Woodstock, NY}
% \acmBooktitle{Woodstock '18: ACM Symposium on Neural Gaze Detection,
%   June 03--05, 2018, Woodstock, NY}
% \acmPrice{15.00}
% \acmISBN{978-1-4503-XXXX-X/18/06}


%%
%% Submission ID.
%% Use this when submitting an article to a sponsored event. You'll
%% receive a unique submission ID from the organizers
%% of the event, and this ID should be used as the parameter to this command.
%%\acmSubmissionID{123-A56-BU3}

%%
%% The majority of ACM publications use numbered citations and
%% references.  The command \citestyle{authoryear} switches to the
%% "author year" style.
%%
%% If you are preparing content for an event
%% sponsored by ACM SIGGRAPH, you must use the "author year" style of
%% citations and references.
%% Uncommenting
%% the next command will enable that style.
%%\citestyle{acmauthoryear}

%%
%% end of the preamble, start of the body of the document source.
\begin{document}

%%
%% The "title" command has an optional parameter,
%% allowing the author to define a "short title" to be used in page headers.
\title[Understanding the Impact of COVID-19 on Health and Wellbeing]{Understanding the Impact of COVID-19 on Health and Wellbeing: Capturing Perspectives from Computer Science Practitioners}

%%
%% The "author" command and its associated commands are used to define
%% the authors and their affiliations.
%% Of note is the shared affiliation of the first two authors, and the
%% "authornote" and "authornotemark" commands
%% used to denote shared contribution to the research.
\author{Tom Crick}
%\authornote{Both authors contributed equally to this research.}
\email{thomas.crick@swansea.ac.uk}
\orcid{0000-0001-5196-9389}
%\author{}
%\authornotemark[1]
\affiliation{%
  \institution{Swansea University}
  \city{Swansea}
  \country{UK}
}

\author{Cathryn Knight}
\email{cathryn.knight@swansea.ac.uk}
\orcid{0000-0002-7574-3090}
\affiliation{%
  \institution{Swansea University}
  \city{Swansea}
  \country{UK}}
%\email{larst@affiliation.org}

\author{Richard Watermeyer}
\email{richard.watermeyer@bristol.ac.uk}
\orcid{0000-0002-2365-3771}
\affiliation{%
  \institution{University of Bristol}
  \city{Bristol}
  \country{UK}
}

%%
%% By default, the full list of authors will be used in the page
%% headers. Often, this list is too long, and will overlap
%% other information printed in the page headers. This command allows
%% the author to define a more concise list
%% of authors' names for this purpose.
\renewcommand{\shortauthors}{Crick, et al.}

%%
%% The abstract is a short summary of the work to be presented in the
%% article.
\begin{abstract}
From March 2020, the COVID-19 pandemic imposed ``emergency remote
teaching'' across education globally, leading to the closure of
institutions across all settings. The resulting shift to online
learning, teaching and assessment (LT\&A) has placed significant
challenges on practitioners, especially their mental health and
wellbeing. Building on previous work, this poster presents preliminary
results drawn from international computer science practitioners
(n=779) extracted from a wider sample of higher education academics
(N=2,628). We highlight widespread concerns relating to transitioning
to remote online working; deprioritisation of research; and wider
impact on marginalised communities within the discipline. These
preliminary results offers valuable insight into the impact of
COVID-19 on the health and wellbeing of computer science
practitioners, especially as we start to move towards a new post-COVID
(ab)normal.
\end{abstract}

%%
%% The code below is generated by the tool at http://dl.acm.org/ccs.cfm.
%% Please copy and paste the code instead of the example below.
%%
\begin{CCSXML}
<ccs2012>
<concept>
<concept_id>10003456.10003457.10003527</concept_id>
<concept_desc>Social and professional topics~Computing education</concept_desc>
<concept_significance>500</concept_significance>
</concept>
<concept>
<concept_id>10003456.10010927</concept_id>
<concept_desc>Social and professional topics~User characteristics</concept_desc>
<concept_significance>500</concept_significance>
</concept>
<concept>
<concept_id>10002944.10011123.10010912</concept_id>
<concept_desc>General and reference~Empirical studies</concept_desc>
<concept_significance>500</concept_significance>
</concept>
</ccs2012>
\end{CCSXML}

\ccsdesc[500]{Social and professional topics~Computing education}
\ccsdesc[500]{Social and professional topics~User characteristics}
\ccsdesc[500]{General and reference~Empirical studies}

%%
%% Keywords. The author(s) should pick words that accurately describe
%% the work being presented. Separate the keywords with commas.
\keywords{COVID-19, emergency remote teaching, practitioner
perceptions, pedagogy, assessment, curriculum, computer science
education}


%%
%% This command processes the author and affiliation and title
%% information and builds the first part of the formatted document.
\maketitle

\section{Introduction}

% The long-term impact of the COVID-19 global pandemic is still
% incalculable; it has affected, and continues to affect, profound
% social suffering, significant health and wellbeing impacts, widespread
% cultural disruption, and deep economic hardship. While indiscriminate
% in terms of whom it infects, it has largely punished society’s most
% vulnerable and less fortunate~\cite{vanlancker+parolin:2020}; even
% with global vaccination efforts, it appears that the virus may have to
% be tolerated on an longer-term basis~\cite{kissler-et-al:2020}.

The impact of the COVID-19 pandemic on education systems globally --
across all settings and contexts -- has been
profound~\cite{unescocovidedu:2021}, presenting significant challenges
for learning, teaching and assessment
(LT\&A)~\cite{itnowdigedu:2021}. It is clear that the academic
discipline of computer science has much to offer to address the
breadth of societal challenges resulting from the COVID-19
pandemic~\cite{cerf:2020}; however, there has been less focus to date
on what this means for CS education, and especially on the impact on
CS practitioners.

\section{Related Work}

Our current research builds on previous work looking at CS education
practitioners, conducted in the immediate aftermath of``lockdowns''
and closures of educational institutions, both in the UK and
internationally~\cite{crick-et-al:ukicer2020,crick-et-alposter:sigcse2021,crick-et-al:educon2021,wg1:iticse2021}. This
CS-specific work has resulted from larger empirical studies looking at
the wider impact of COVID-19 on education and practitioners,
especially in higher education, and the ``digital disruption'' from
the shift to
online~\cite{watermeyer-et-al:he2021,shankar-et-al:ies2021,watermeyer-et-al:bjse2021}.

\section{Health and Wellbeing of Computer Science Practitioners}

The ongoing consequences for practitioner wellbeing,
work-life balance and the wider risks associated with educational
establishments ``returning to normal'' after
COVID-19~\cite{yamey+walensky:2020} are only just starting to be addressed
and evaluated in the academic trade and policy literature. There have
been few studies examining the experiences of academic practitioners
vis-à-vis mental health and wellbeing, their hopes and
concerns for the future, or on their experiences of academic
leadership during the pandemic.

We introduce preliminary results drawn from international computer
science practitioners (n=779) extracted from a wider international
survey of higher education academics (N=2,628). It examines the
COVID-19 pandemic in educational establishments, encompassing
teaching, research, physical and emotional wellbeing, as well as their
perceptions of the response of their leaders. An online survey was
designed and distributed via Qualtrics; it was launched on 19 June
2020 and remained open for two months. The survey consisted of
demographic questions to determine relevant information apropos
respondents’ educational setting, role and individual characteristics,
and a series of Likert scale closed-ended questions designed to
identify the impact of COVID-19 on their professional
lives. Respondents were also asked open-ended questions designed to
capture more in-depth accounts of their personal experiences, which
were coded and thematically analysed.

\section{Initial Findings}

Analysis from this work is ongoing, but preliminary results from the
international sample of computer science practitioners highlight
widespread concerns relating to transitioning to remote online
working; research productivity and deprioritisation of research; and
work intensification. Related concerns around changes to academics’
working conditions and renewed managerialism; longer-term impacts of
changes in the organisation and structure of the academic role (for
example, the unevenly distributed impact on early-career, female and
ethnic minority academics); as well as acknowledging the policy
prominence of computer science as a discipline (especially for
post-COVID economic recovery). Our survey indicates a consensus of the
adverse effects of universities’ response to the COVID-19 pandemic --
what we have called ``pandemia'' -- enacted on the professional and
personal wellbeing of academic communities.

% This wider work also directly links to recent major
% international changes to CS curricula, qualifications
% and practice (e.g. in the
% UK~\cite{brown-et-al-sigcse2013,wgictreview:2013,brown-et-al-toce2014,crick:iticse2021}), as well as the
% emerging focus on the required skills and infrastructure interventions
% to support the global post-COVID economic
% recovery~\cite{davenport-et-al:educon2020,euparl:2020,mckinsey:2020}.



%%
%% The acknowledgments section is defined using the "acks" environment
%% (and NOT an unnumbered section). This ensures the proper
%% identification of the section in the article metadata, and the
%% consistent spelling of the heading.
% \begin{acks}
% To Robert, for the bagels and explaining CMYK and color spaces.
% \end{acks}

% \appendix

% \section*{Background Information}

% \noindent {\emph{A) Is it your first time presenting at ICER?}} Yes\\

% \noindent {\emph{B) Phase of work at time of presentation will be:}} Study
% completed; initial results\\

% \noindent {\emph{C) My goal in presenting this poster is:}} Solicit feedback
% from the community regarding a CER project; present unpublished results of ongoing work\\

% \noindent Related COVID-19 work can be found here: \url{https://proftomcrick.com/tag/covid-19/}

%%
%% The next two lines define the bibliography style to be used, and
%% the bibliography file.
\bibliographystyle{ACM-Reference-Format}
\bibliography{icer2021}

\end{document}
\endinput
%%
%% End of file `sample-manuscript.tex'.
